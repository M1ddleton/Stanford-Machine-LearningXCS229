\item {\bf Incomplete, Positive-Only Labels}

In this problem we will consider training binary classifiers in situations
where we do not have full access to the labels. In particular, we consider
a scenario, which is not too infrequent in real life, where we have labels
only for a subset of the positive examples. All the negative examples and
the rest of the positive examples are unlabelled.

We formalize the scenario as follows. Let $\{(x^{(i)}, t^{(i)})\}_{i=1}^\nexp$ be a standard dataset of i.i.d distributed examples. Here $x^{(i)}$'s are the inputs/features and $t^{(i)}$ are the labels. Now consider the situation where $t^{(i)}$'s are not observed by us. Instead, we only observe the labels of some of the positive examples. Concretely, we assume that we observe  $y^{(i)}$'s that are generated by
\begin{align*}
& \forall  x, ~~ p(y^{(i)} = 1\mid t^{(i)}=1, x^{(i)}=x) = \alpha, \\
& \forall  x, ~~ p(y^{(i)} = 0 \mid t^{(i)}=1, x^{(i)}=x)  = 1- \alpha\\
& \forall  x, ~~ p(y^{(i)} = 1 \mid t^{(i)}=0, x^{(i)}=x) = 0,\\ 
& \forall  x, ~~ p(y^{(i)} = 0 \mid t^{(i)}=0, x^{(i)}=x) = 1
\end{align*}
where $\alpha \in (0,1)$ is some unknown scalar. In other words, if the unobserved ``true'' label $t^{(i)}$ is 1, then with $\alpha$ chance we observe a label $y^{(i)} = 1$. On the other hand, if the unobserved ``true'' label $t^{(i)} = 0$, then we always observe the label $y^{(i)} = 0$. 

Our final goal in the problem is to construct a binary classifier $h$ of
the true label $t$, with only access to the partial label $y$. In other words,
we want to construct $h$ such that
 $h(x^{(i)}) \approx p(t^{(i)} = 1\mid x^{(i)})$ as closely as
possible, using only $x$ and $y$.

\emph{Real world example: Suppose we maintain a database of proteins which
are involved in transmitting signals across membranes. Every example added to
the database is involved in a signaling process, but there are many proteins
involved in cross-membrane signaling which are missing from the database.
It would be useful to train a classifier to identify proteins that
should be added to the database. In our notation, each example $x^{(i)}$
corresponds to a protein, $y^{(i)} = 1$ if the protein is in the database and
$0$ otherwise, and $t^{(i)} = 1$ if the protein is involved in a cross-membrane
signaling process and thus should be added to the database, and $0$ otherwise.}


For the rest of the question, we will use the dataset and starter code provided in
the following files:
%
\begin{center}
\begin{itemize}
\item	\url{src-incomplete/{train,valid,test}.csv}
\item   \url{src-incomplete/submission.py}
\end{itemize}
\end{center}
%
Each file contains the following columns: $x_1$, $x_2$, $y$, and $t$. As in
Problem 1, there is one example per row. The $y^{(i)}$'s are generated from the process defined above with some unknown $\alpha$.
\textbf{Note: We will be using the logistic regression classifier defined in problem 1 to apply binary classification for problem 2. If you haven't done so already, finish implementing the Logistic Regression class in src-linear/submission/logreg.py.}

\textbf{Code Deliverables}
\begin{itemize}
    \item \texttt{src-incomplete/submission.py}
    \item \texttt{src-incomplete/logreg.py} \textbf{  **Copied over from src-linear/submission/logreg.py}
\end{itemize}


\begin{enumerate}
    \item \points{2a} \textbf{Coding problem: ideal (fully observed) case}

First we will consider the hypothetical (and uninteresting) case, where we have access to the true
$t$-labels for training. In \texttt{src-incomplete/submission.py}, write a logistic
regression classifier that uses $x_1$ and $x_2$ as input features, and train it
using the $t$-labels. More specifically you will implement the |fully_observed_predictions| function.
We will ignore the $y$-labels for this part. Output the
trained model's predictions on the \textbf{test set} to the file specified in the code.

The output plot should look similar to the following (no plot submission is required):
\begin{figure}[H]
	\centering
	\vspace{2mm}
	\includegraphics[width=0.5\linewidth]{02-posonly/posonly_true_pred.png}
    \caption{Separating hyperplane for logistic regression on training set using fully observed predictions (Note: This is for reference only.  You are not required to submit a plot.)}
\end{figure}

    \item \points{2b} \textbf{Coding problem: The naive method on partial labels}

We now consider the case where the $t$-labels are unavailable, so you only have
access to the $y$-labels at training time. Extend your code in
\texttt{src-incomplete/submission.py} to re-train the classifier (still using $x_1$ and
$x_2$ as input features), but using the $y$-labels only. More specifically, you will implement
the |naive_partial_labels_predictions| function. Output the predictions
on the \textbf{test set} to the appropriate file (as described in the code comments).

Note that the algorithm should learn a function $h(\cdot)$ that approximately predicts the probability $p(y^{(i)}=1\mid x^{(i)})$. Also note that we expect it to perform poorly on predicting the probability of interest, namely $p(t^{(i)}=1\mid x^{(i)})$.

The output plot should look similar to the following (no plot submission is required):
\begin{figure}[H]
	\centering
	\vspace{2mm}
	\includegraphics[width=0.5\linewidth]{02-posonly/posonly_naive_pred.png}
    \caption{Separating hyperplane for logistic regression on training set using naive partially observed labels (Note: This is for reference only.  You are not required to submit a plot.)}
\end{figure}



    \input{02-posonly/03-bayes-warm-up}

    \input{02-posonly/04-constant}

    \input{02-posonly/05-estimate-alpha}

    \item \points{2f} \textbf{Coding problem.}

Using the validation set, estimate the constant $\alpha$ by averaging your
classifier's predictions over all labeled examples in the validation set:\footnote{There is a reason to use the validation set, instead of the training set, to estimate the $\alpha$. However, for the purpose of this question, we sweep the subtlety here under the rug, and you don't need to understand the difference between the two for this question. } 
%
\begin{equation*}
  \alpha \approx \frac{1}{\mid V_{+} \mid}\sum_{x^{(i)}\in V_{+}} h(x^{(i)}).
\end{equation*}
%
Add code in \texttt{src-incomplete/submission.py} to rescale your
 predictions $h(y^{(i)}=1\mid x^{(i)})$ of the classifier that is obtained from part b,  using the equation~\eqref{eqn:3} obtained in part (d) and using the estimated value for $\alpha$. 
More specifically implement the |find_alpha_and_plot_correction| function.

 The output plot should look similar to the following (no plot submission is required):
 \begin{figure}[H]
   \centering
   \vspace{2mm}
   \includegraphics[width=0.5\linewidth]{02-posonly/posonly_adjusted_pred.png}
     \caption{Separating hyperplane for logistic regression on training set using using corrected alpha value from validation set (Note: This is for reference only.  You are not required to submit a plot.)}
 \end{figure}


\end{enumerate}

\textbf{Remark}: We saw that the true probability $p(t\mid x)$ was only a
constant factor away from $p(y\mid x)$. This means, if our task is to only rank
examples (\emph{i.e.} sort them) in a particular order (e.g, sort the proteins
in order of being most likely to be involved in transmitting signals across
membranes), then in fact we do not even need to estimate $\alpha$. The rank
based on $p(y\mid x)$ will agree with the rank based on $p(t\mid x)$.