\item {\bf Implicit Regularization}

Recall that in the overparameterized regime (where the number of parameters is larger than the number of samples), typically there are infinitely many solutions that can fit the training dataset perfectly, and many of them cannot generalize well (that is, they have large validation errors). However, in many cases, the particular optimizer we use (e.g., GD, SGD with particular learning rates, batch sizes, noise, etc.) tends to find solutions that generalize well. This phenomenon is called implicit regularization effect (also known as algorithmic regularization or implicit bias). 

In this problem, we will look at the implicit regularization effect on two toy examples in the overparameterized regime: linear regression and a quadratically parameterized model. For linear regression, we will show that gradient descent with zero initialization will always find the minimum norm solution (instead of an arbitrary solution that fits the training data), and in practice, the minimum norm solution tends to generalize well. For a quadratically parameterized model, we will show that initialization and batch size also affect generalization.

\begin{enumerate}
    \item \points{4a}  Suppose we have a dataset $\{(x^{(i)}, y^{(i)});i=1,\cdots,n\}$ where $x^{(i)}\in \R^{d}$ and $y^{(i)}\in \R$ for all $1\le i\le n.$ We assume the dataset is generated by a linear model without noise. That is, there is a vector $\beta^\star\in\R^{d}$ such that $y^{(i)}=(\beta^\star)^\top x^{(i)}$ for all $1\le i\le n$. Let $X\in \R^{n\times d}$ be the matrix representing the inputs (i.e., the $i$-th row of $X$ corresponds to $x^{(i)})$) and $\vec{y}\in\R^{n}$ the vector representing the labels (i.e., the $i$-th row of $\vec{y}$ corresponds to $y^{(i)})$):
$$
	X=
\begin{bmatrix}
	- & x^{(1)} & - \\
	- & x^{(2)} & - \\
	\vdots & \vdots & \vdots\\
	- & x^{(n)} & - 
\end{bmatrix},\qquad
\vec{y}=
\begin{bmatrix}
y^{(1)} \\
y^{(2)}\\
\vdots\\
y^{(n)}
\end{bmatrix}.
$$
Then in matrix form, we can write $\vec{y}=X\beta^\star.$
We assume that the number of examples is less than the number of parameters (that is, $n<d$).

We use the least-squares cost function to train a linear model:
\begin{equation}\label{equ:mse}
	J(\beta)=\frac{1}{2n}\|X\beta-\vec{y}\|_2^2.
\end{equation}

In this sub-question, we characterize the family of global minimizers to Eq.~\eqref{equ:mse}. We assume that $X X^\top\in \R^{n\times n}$ is an invertible matrix. \textbf{Prove that} $\beta$ achieves zero cost in Eq.~\eqref{equ:mse} if and only if 
\begin{equation}\label{equ:ir2}
	\beta=X^\top (XX^\top)^{-1}\vec{y}+\zeta
\end{equation} for some $\zeta$ in the subspace orthogonal to all the data (that is, for some $\zeta$ such that $\zeta^\top x^{(i)}=0,\forall 1\le i\le n.$) 

Note that this implies that there is an infinite number of $\beta$'s such that Eq.~\eqref{equ:mse} is minimized. We also note that $X^\top (XX^\top)^{-1}$ is the pseudo-inverse of $X$, but you don't necessarily need this fact for the proof.


    
    \item \points{4b} We still work with the setup of part (a). Among the infinitely many optimal solutions of Eq.~\eqref{equ:mse}, we consider the \textit{minimum norm} solution. Let $\rho=X^\top (X X^\top)^{-1}\vec{y}$. In the setting of (a), \textbf{prove that} for any $\beta$ such that $J(\beta)=0$, $\|\rho\|_2\le \|\beta\|_2.$ In other words, $\rho$ is the minimum norm solution.

\emph{Hint:} As a intermediate step, you can prove that for any $\beta$ in the form of Eq.~\eqref{equ:ir2}, $$\|\beta\|_2^2=\|\rho\|_2^2+\|\zeta\|_2^2.$$
    
    \item \points{4c} \textbf{Minimum norm solution generalizes well}

For this sub-question, we still work with the setup of parts (a) and (b). We use the following datasets:
\begin{center}
	\texttt{src-implicitreg/ir1\_train.csv, ir1\_valid.csv}
\end{center}
Each file contains $d+1$ columns. The first $d$ columns in the $i$-th row represents $x^{(i)}$, and the last column represents $y^{(i)}.$ In this sub-question, we use $d=200$ and $n=40$.

Using the formula in sub-question (b), \textbf{compute} the minimum norm solution using the training dataset. Then, \textbf{generate} three other different solutions with zero costs and different norms using the formula in sub-question (a).
The starter code is in \texttt{src-implicitreg/submission.py} and you should implement the \texttt{get\_minimum\_norm\_solution} and \texttt{get\_different\_n\_solutions} functions.  
Your generated plot should demonstrate that the minimum norm solution generalizes well and should be similar to the following plot:

\begin{figure}[H]
	\centering
	\includegraphics[width=.5\linewidth]{04-implicitreg/implicitreg_linear.png}
\end{figure}
	
	\item \points{4d} 
For this sub-question, we work with the setup of part (a) and (b). In this sub-question, you will prove that the gradient descent algorithm with \emph{zero initialization} always converges to the minimum norm solution. Let $\beta^{(t)}$ be the parameters found by the GD algorithm at time step $t$. Recall that at step $t$, the gradient descent algorithm update the parameters in the following way
\begin{equation}
	\beta^{(t)}=\beta^{(t-1)}-\eta\nabla J(\beta^{(t-1)})=\beta^{(t-1)}-\frac{\eta}{n} X^\top (X\beta^{(t-1)}-\vec{y}).
\end{equation}
As in sub-question (a), we also assume $X X^\top$ is an invertible matrix. \textbf{Prove} that if the GD algorithm  with zero initialization converges to a solution $\hat{\beta}$ satisfying $J(\hat{\beta})=0$, then $\hat{\beta}=X^\top(XX^\top)^{-1}\vec{y}=\rho,$, that is, $\hat{\beta}$ is the minimum norm solution.

\emph{Hint:} As a first step, you can prove by induction that if we start with zero initialization, $\beta^{(t)}$ will always be a linear combination of $\{x^{(1)}, x^{(2)}, \cdots, x^{(n)}\}$ for any $t\ge 0.$ Then, for any $t\ge 0$, you can write $\beta^{(t)}=X^\top v^{(t)}$ for some $v^{(t)}\in \R^{n}.$ As a second step, you can prove that if $\hat{\beta}=X^\top v^{(t)}$ for some $v^{(t)}$ and $J(\hat{\beta})=0$, then we have $\hat{\beta}=\rho.$

You don't necessarily have to follow the steps in this hint. But if you use the hint, you need to prove the statements in the hint.


	\item \points{4e}
In the following sub-questions, we consider a slightly more complicated model called quadratically parameterized model. A quadratically parameterized model has two sets of parameters $\theta,\phi\in \R^{d}$. Given a $d$-dimensional input $x\in \R^{d}$, the output of the model is 
\begin{align}
	f_{\theta,\phi}(x)=\sum_{k=1}^{d}\theta_k^2 x_k - \sum_{k=1}^{d}\phi_k^2 x_k.\label{eqn:1}
\end{align}
Note that $f_{\theta,\phi}(x)$ is linear in its input $x$, but non-linear in its parameters $\theta,\phi.$ Thus, if the goal was to learn the function, one should simply just re-parameterize it with a linear model and use linear regression. However, here we insist on using the parameterization above in Eq.~\eqref{eqn:1} in order to study the implicit regularization effect in models that are nonlinear in the parameters.

\emph{Notations:} To simplify the equations, we define the following notations. For a vector $v\in \R^{d},$ let $v^{\odot 2}$ be its element-wise square (that is, $v^{\odot 2}$ is the vector $[v_1^2, v_2^2,\cdots, v_d^2]\in \R^{d}.$)  For two vectors $v,w\in \R^{d},$ let $v\odot w$ be their element-wise product (that is, $v\odot w$ is the vector $[v_1w_1,v_2w_2,\cdots,v_dw_d]\in \R^{d}.$) Then our model can be written as 
\begin{align}
	f_{\theta,\phi}(x)=x^\top (\theta^{\odot 2} -\phi^{\odot 2}).
\end{align}

Suppose we have a dataset $\{(x^{(i)}, y^{(i)});i=1,\cdots,n\}$ where $x^{(i)}\in \R^{d}$ and $y^{(i)}\in \R$ for all $1\le i\le n,$ and $$y^{(i)}=(x^{(i)})^\top ((\theta^\star)^{\odot 2}- (\phi^\star)^{\odot 2})$$ for some $\theta^\star, \phi^\star\in \R^{d}$.
Similarly, we use $X\in \R^{n\times d}$ and $\vec{y}\in \R^{n}$ to denote the matrix/vector representing the inputs/labels respectively:
$$
X=
\begin{bmatrix}
	- & x^{(1)} & - \\
	- & x^{(2)} & - \\
	\vdots & \vdots & \vdots\\
	- & x^{(n)} & - 
\end{bmatrix},\qquad
\vec{y}=
\begin{bmatrix}
	y^{(1)} \\
	y^{(2)}\\
	\vdots\\
	y^{(n)}
\end{bmatrix}.
$$
Let $J(\theta,\phi)=\frac{1}{4n}\sum_{i=1}^{n}(f_{\theta,\phi}(x^{(i)})-y^{(i)})^2$ be the cost function.

First, when $n<d$ and $XX^\top$ is invertible, \textbf{prove} that there exists infinitely many optimal solutions with zero cost.

\emph{Hint:} Find a mapping between the parameter $\beta$ in linear model and the parameter $\theta,\phi$ in quadratically parameterized model. Then use the conclusion in sub-question (a).

	\item \points{4f} \textbf{Implicit regularization of initialization}

We still work with the setup in part (e).
For this sub-question, we use the following datasets:
\begin{center}
	\texttt{src-implicitreg/ir2\_train.csv, ir2\_valid.csv}
\end{center}
Each file contains $d+1$ columns. The first $d$ columns in the $i$-th row represents $x^{(i)}$, and the last column represents $y^{(i)}.$ In this sub-question, we use $d=200$ and $n=40$.

First of all, the gradient of the loss has the following form:
\begin{align}
	\nabla_\theta J(\theta,\phi)&=\frac{1}{n}\sum_{i=1}^{n}((x^{(i)})^\top (\theta^{\odot 2} -\phi^{\odot 2})-y^{(i)})(\theta\odot x^{(i)}),\\
	\nabla_\phi J(\theta,\phi)&=-\frac{1}{n}\sum_{i=1}^{n}((x^{(i)})^\top (\theta^{\odot 2} -\phi^{\odot 2})-y^{(i)})(\phi\odot x^{(i)}).
\end{align}
You don't need to prove these two equations. They can be verified directly using the chain rule.

Using the formula above, run gradient descent with initialization $\theta=\alpha \mathbf{1}, \phi=\alpha\mathbf{1}$ with $\alpha\in \{0.1, 0.03, 0.01\}$ (where $\mathbf{1}=[1,1,\cdots,1]\in\R^{d}$ is the all-1's vector) and learning rate $0.08$. We provide the starter code in \texttt{src-implicitreg/submission.py} and you should implement the \texttt{QP.gradient} and \texttt{QP.train\_GD} functions. Your generated plot should be looking like the following:

\begin{figure}[H]
	\centering
	\includegraphics[width=.7\linewidth]{04-implicitreg/implicitreg_quadratic_initialization.png}
\end{figure}

\textit{Remark:} Your plot is expected to demonstrate that the initialization plays an important role in the generalization performance---different initialization can lead to different global minimizers with different generalization performance. In other words, the initialization has an implicit regularization effect. 

	\item \points{4g} \textbf{Implicit regularization of initialization analysis}

\textbf{Answer} the following two questions based on the plot in part (f):

\begin{itemize}
    \item Which models can fit the training set?
    
    \item Which initialization achieves the best validation error? 
\end{itemize}





	\item \points{4h} \textbf{Implicit regularization of batch size}

We still work with the setup in part (e). For this sub-question, we use the same dataset and starter code as in sub-question (f). We will show that the noise in the training process also induces implicit regularization. In particular, the noise introduced by \emph{stochastic} gradient descent in this case helps generalization. \textbf{Implement} the SGD algorithm in the \texttt{QP.train\_SGD} function and run it with batch size $\{1, 5, 40\}$, learning rate $0.08$, and initialization $\alpha=0.1$. For simplicity, the code for selecting a batch of examples is already provided in the starter code.

Your generated plot should like like the following:

\begin{figure}[H]
    \centering
    \includegraphics[width=.7\linewidth]{04-implicitreg/implicitreg_quadratic_batchsize.png}
\end{figure}

The plot shows that the stochasticity in the training process is also an important factor in the generalization performance --- in our setting, SGD finds a solution that generalizes better. In fact, a conjecture is that stochasticity in the optimization process (such as the noise introduced by a small batch size) helps the optimizer to find a solution that generalizes better. This conjecture can be proved in some simplified cases, such as the quadratically parameterized model in this sub-question (adapted from the paper \href{https://arxiv.org/abs/2006.08680}{HaoChen et al., 2020}), and can be observed empirically in many other cases.

\textbf{Files to Submit: }\texttt{src-implicitreg/submission.py}

	\item \points{4i} \textbf{GD vs SGD}

\textbf{Compare} the results with those in sub-question (f) with the same initialization. Does SGD find a better solution?

\end{enumerate}
