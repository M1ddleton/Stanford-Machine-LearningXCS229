% This contents of this file will be inserted into the _Solutions version of the
% output tex document.  Here's an example:

% If assignment with subquestion (1.a) requires a written response, you will
% find the following flag within this document: <SCPD_SUBMISSION_TAG>_1a
% In this example, you would insert the LaTeX for your solution to (1.a) between
% the <SCPD_SUBMISSION_TAG>_1a flags.  If you also constrain your answer between the
% START_CODE_HERE and END_CODE_HERE flags, your LaTeX will be styled as a
% solution within the final document.

% Please do not use the '<SCPD_SUBMISSION_TAG>' character anywhere within your code.  As expected,
% that will confuse the regular expressions we use to identify your solution.
\def\assignmentnum{5 }
\def\assignmenttitle{XCS229 Problem Set \assignmentnum}

\documentclass{article}
\usepackage[top = 1.0in]{geometry}

\usepackage{graphicx}

\usepackage[utf8]{inputenc}
\usepackage{listings}
\usepackage[dvipsnames]{xcolor}
\usepackage{bm}
\usepackage{algorithm}
\usepackage{algpseudocode}
\usepackage{framed}

\definecolor{codegreen}{rgb}{0,0.6,0}
\definecolor{codegray}{rgb}{0.5,0.5,0.5}
\definecolor{codepurple}{rgb}{0.58,0,0.82}
\definecolor{backcolour}{rgb}{0.95,0.95,0.92}

\lstdefinestyle{mystyle}{
    backgroundcolor=\color{backcolour},   
    commentstyle=\color{codegreen},
    keywordstyle=\color{magenta},
    stringstyle=\color{codepurple},
    basicstyle=\ttfamily\footnotesize,
    breakatwhitespace=false,         
    breaklines=true,                 
    captionpos=b,                    
    keepspaces=true,                 
    numbersep=5pt,                  
    showspaces=false,                
    showstringspaces=false,
    showtabs=false,                  
    tabsize=2
}

\lstset{style=mystyle}

\newcommand{\di}{{d}}
\newcommand{\nexp}{{n}}
\newcommand{\nf}{{p}}
\newcommand{\vcd}{{\textbf{D}}}
\newcommand{\Int}{\mathbb{Z}}
\newcommand\bb{\ensuremath{\mathbf{b}}}
\newcommand\bs{\ensuremath{\mathbf{s}}}
\newcommand\bp{\ensuremath{\mathbf{p}}}
\newcommand{\relu} { \operatorname{ReLU} }
\newcommand{\smx} { \operatorname{softmax} }
\newcommand\bx{\ensuremath{\mathbf{x}}}
\newcommand\bh{\ensuremath{\mathbf{h}}}
\newcommand\bc{\ensuremath{\mathbf{c}}}
\newcommand\bW{\ensuremath{\mathbf{W}}}
\newcommand\by{\ensuremath{\mathbf{y}}}
\newcommand\bo{\ensuremath{\mathbf{o}}}
\newcommand\be{\ensuremath{\mathbf{e}}}
\newcommand\ba{\ensuremath{\mathbf{a}}}
\newcommand\bu{\ensuremath{\mathbf{u}}}
\newcommand\bv{\ensuremath{\mathbf{v}}}
\newcommand\bP{\ensuremath{\mathbf{P}}}
\newcommand\bg{\ensuremath{\mathbf{g}}}
\newcommand\bX{\ensuremath{\mathbf{X}}}
% real numbers R symbol
\newcommand{\Real}{\mathbb{R}}

% encoder hidden
\newcommand{\henc}{\bh^{\text{enc}}}
\newcommand{\hencfw}[1]{\overrightarrow{\henc_{#1}}}
\newcommand{\hencbw}[1]{\overleftarrow{\henc_{#1}}}

% encoder cell
\newcommand{\cenc}{\bc^{\text{enc}}}
\newcommand{\cencfw}[1]{\overrightarrow{\cenc_{#1}}}
\newcommand{\cencbw}[1]{\overleftarrow{\cenc_{#1}}}

% decoder hidden
\newcommand{\hdec}{\bh^{\text{dec}}}

% decoder cell
\newcommand{\cdec}{\bc^{\text{dec}}}

\usepackage[hyperfootnotes=false]{hyperref}
\hypersetup{
  colorlinks=true,
  linkcolor = blue,
  urlcolor  = blue,
  citecolor = blue,
  anchorcolor = blue,
  pdfborderstyle={/S/U/W 1}
}
\usepackage{nccmath}
\usepackage{mathtools}
\usepackage{graphicx,caption}
\usepackage[shortlabels]{enumitem}
\usepackage{epstopdf,subcaption}
\usepackage{psfrag}
\usepackage{amsmath,amssymb,epsf}
\usepackage{verbatim}
\usepackage{cancel}
\usepackage{color,soul}
\usepackage{bbm}
\usepackage{listings}
\usepackage{setspace}
\usepackage{float}
\definecolor{Code}{rgb}{0,0,0}
\definecolor{Decorators}{rgb}{0.5,0.5,0.5}
\definecolor{Numbers}{rgb}{0.5,0,0}
\definecolor{MatchingBrackets}{rgb}{0.25,0.5,0.5}
\definecolor{Keywords}{rgb}{0,0,1}
\definecolor{self}{rgb}{0,0,0}
\definecolor{Strings}{rgb}{0,0.63,0}
\definecolor{Comments}{rgb}{0,0.63,1}
\definecolor{Backquotes}{rgb}{0,0,0}
\definecolor{Classname}{rgb}{0,0,0}
\definecolor{FunctionName}{rgb}{0,0,0}
\definecolor{Operators}{rgb}{0,0,0}
\definecolor{Background}{rgb}{0.98,0.98,0.98}
\lstdefinelanguage{Python}{
    numbers=left,
    numberstyle=\footnotesize,
    numbersep=1em,
    xleftmargin=1em,
    framextopmargin=2em,
    framexbottommargin=2em,
    showspaces=false,
    showtabs=false,
    showstringspaces=false,
    frame=l,
    tabsize=4,
    % Basic
    basicstyle=\ttfamily\footnotesize\setstretch{1},
    backgroundcolor=\color{Background},
    % Comments
    commentstyle=\color{Comments}\slshape,
    % Strings
    stringstyle=\color{Strings},
    morecomment=[s][\color{Strings}]{"""}{"""},
    morecomment=[s][\color{Strings}]{'''}{'''},
    % keywords
    morekeywords={import,from,class,def,for,while,if,is,in,elif,else,not,and,or
    ,print,break,continue,return,True,False,None,access,as,,del,except,exec
    ,finally,global,import,lambda,pass,print,raise,try,assert},
    keywordstyle={\color{Keywords}\bfseries},
    % additional keywords
    morekeywords={[2]@invariant},
    keywordstyle={[2]\color{Decorators}\slshape},
    emph={self},
    emphstyle={\color{self}\slshape},
%
}
\lstMakeShortInline|

\pagestyle{empty} \addtolength{\textwidth}{1.0in}
\addtolength{\textheight}{0.5in}
\addtolength{\oddsidemargin}{-0.5in}
\addtolength{\evensidemargin}{-0.5in}
\newcommand{\ruleskip}{\bigskip\hrule\bigskip}
\newcommand{\nodify}[1]{{\sc #1}}
\newenvironment{answer}{\sf \begingroup\color{ForestGreen}}{\endgroup}%

\setlist[itemize]{itemsep=2pt, topsep=0pt}
\setlist[enumerate]{itemsep=6pt, topsep=0pt}

\setlength{\parindent}{0pt}
\setlength{\parskip}{4pt}
\setlist[enumerate]{parsep=4pt}
\setlength{\unitlength}{1cm}

\renewcommand{\Re}{{\mathbb R}}
\newcommand{\R}{\mathbb{R}}
\newcommand{\what}[1]{\widehat{#1}}

\renewcommand{\comment}[1]{}
\newcommand{\mc}[1]{\mathcal{#1}}
\newcommand{\half}{\frac{1}{2}}

\DeclareMathOperator*{\argmin}{arg\,min}

\def\KL{D_{KL}}
\def\xsi{x^{(i)}}
\def\ysi{y^{(i)}}
\def\zsi{z^{(i)}}
\def\E{\mathbb{E}}
\def\calN{\mathcal{N}}
\def\calD{\mathcal{D}}
\def\slack{\url{http://xcs229-scpd.slack.com/}}
\def\zipscriptalt{\texttt{python zip\_submission.py}}
\DeclarePairedDelimiter\abs{\lvert}{\rvert}%
 
\usepackage{bbding}
\usepackage{pifont}
\usepackage{wasysym}
\usepackage{amssymb}
\usepackage{framed}
\usepackage{scrextend}

\newcommand{\alns}[1] {
	\begin{align*} #1 \end{align*}
}

\newcommand{\pd}[2] {
 \frac{\partial #1}{\partial #2}
}
\renewcommand{\Re} { \mathbb{R} }
\newcommand{\btx} { \mathbf{\tilde{x}} }
\newcommand{\bth} { \mathbf{\tilde{h}} }
\newcommand{\sigmoid} { \operatorname{\sigma} }
\newcommand{\CE} { \operatorname{CE} }
\newcommand{\byt} { \hat{\by} }
\newcommand{\yt} { \hat{y} }

\newcommand{\oft}[1]{^{(#1)}}
\newcommand{\fone}{\ensuremath{F_1}}

\newcommand{\ac}[1]{ {\color{red} \textbf{AC:} #1} }
\newcommand{\ner}[1]{\textbf{\color{blue} #1}}
\begin{document}
\pagestyle{myheadings} \markboth{}{\assignmenttitle}

% <SCPD_SUBMISSION_TAG>_entire_submission

This handout includes space for every question that requires a written response.
Please feel free to use it to handwrite your solutions (legibly, please).  If
you choose to typeset your solutions, the |README.md| for this assignment includes
instructions to regenerate this handout with your typeset \LaTeX{} solutions.
\ruleskip

\LARGE
1.
\normalsize

% <SCPD_SUBMISSION_TAG>_1a
  \begin{answer}
      % ### START CODE HERE ###
      % ### END CODE HERE ###
  \end{answer}
% <SCPD_SUBMISSION_TAG>_1a
\clearpage

\LARGE
2.a
\normalsize

% <SCPD_SUBMISSION_TAG>_2a
  \begin{answer}
    \begin{align*}
      \ell(\theta^{(t+1)})
      &= \alpha \ell_\text{sup}(\theta^{(t+1)}) + \ell_\text{unsup}(\theta^{(t+1)})
          &\text{Definition} \\
      &\ge \alpha \ell_\text{sup}(\theta^{(t+1)}) + \sum_{i=1}^\nexp\sum_{\zsi}Q_i^{(t)}(\zsi)\log\frac{p(\xsi,\zsi;\theta^{(t+1)})}{Q_i^{(t)}(\zsi)}
          &\text{Jensen's inequality} \\
      &\ge \\
      % ### START CODE HERE ###
      % ### END CODE HERE ###
    \end{align*}
  \end{answer}
% <SCPD_SUBMISSION_TAG>_2a
\clearpage


\LARGE
2.b
\normalsize

% <SCPD_SUBMISSION_TAG>_2b
  \begin{answer}
  % ### START CODE HERE ###
  % ### END CODE HERE ###
  \end{answer}
% <SCPD_SUBMISSION_TAG>_2b
\clearpage

\LARGE
2.c
\normalsize

% <SCPD_SUBMISSION_TAG>_2c
  \begin{answer}
    List the parameters which need to be re-estimated in the M-step:\\\\\\

    \allowdisplaybreaks

    In order to simplify derivation, it is useful to denote $$w_j^{(i)} = Q^{(t)}_i(\zsi=j),$$ and $$\tilde{w}_j^{(i)} = \begin{cases} \alpha & \tilde{z}^{(i)} = j \\ 0 & \text{ otherwise.} \end{cases}$$

    We further denote $S = \Sigma^{-1}$, and note that because of chain rule of calculus, $\nabla_S\ell = 0 \Rightarrow \nabla_\Sigma \ell = 0$. So we choose to rewrite the M-step in terms of $S$ and maximize it w.r.t $S$, and re-express the resulting solution back in terms of $\Sigma$.

    Based on this, the M-step becomes:
    \begin{align*}
    \phi^{(t+1)}, \mu^{(t+1)}, S^{(t+1)} &=  \arg\max_{\phi,\mu,S} \sum_{i=1}^\nexp \sum_{j=1}^k Q_i^{(t)}(\zsi) \log \frac{p(\xsi,\zsi;\phi,\mu,S)}{Q_i^{(t)}(\zsi)} + \alpha \sum_{i=1}^{\tilde{\nexp}} \log p(\tilde{\xsi}, \tilde{\zsi}; \phi, \mu, S)\\
    &=\\
    % ### START CODE HERE ###
    % ### END CODE HERE ###
    \end{align*}

    Now, calculate the update steps by maximizing the expression within the argmax for each parameter (We will do the first for you).

    ${\mathbf \phi_j}$: We construct the Lagrangian including the constraint that $\sum_{j=1}^k \phi_j = 1$, and absorbing all irrelevant terms into constant $C$:
    \begin{align*}
    \mathcal{L}(\phi, \beta) &= C + \sum_{i=1}^\nexp\sum_{j=1}^k w^{(i)}_j \log \phi_j + \sum_{i=1}^{\tilde{\nexp}}\sum_{j=1}^k \tilde{w}^{(i)}_j \log \phi_j + \beta\left(\sum_{j=1}^k \phi_j - 1\right) \\
    \nabla_{\phi_j}\mathcal{L}(\phi, \beta) &=  \sum_{i=1}^\nexp w^{(i)}_j\frac{1}{\phi_j} + \sum_{i=1}^{\tilde{\nexp}} \tilde{w}^{(i)}_j\frac{1}{\phi_j} + \beta = 0 \\
    &\Rightarrow \phi_j = \frac{\sum_{i=1}^\nexp w^{(i)}_j + \sum_{i=1}^{\tilde{\nexp}} \tilde{w}^{(i)}_j}{-\beta} \\
    \nabla_\beta\mathcal{L}(\phi,\beta) &= \sum_{j=1}^k \phi_j -1 = 0 \\
    &\Rightarrow \sum_{j=1}^k \frac{\sum_{i=1}^\nexp w^{(i)}_j + \sum_{i=1}^{\tilde{\nexp}} \tilde{w}^{(i)}_j}{-\beta} = 1 \\
    &\Rightarrow -\beta = \sum_{j=1}^k \left(\sum_{i=1}^\nexp w^{(i)}_j + \sum_{i=1}^{\tilde{\nexp}} \tilde{w}^{(i)}_j\right)  \\
    \Rightarrow \phi_j^{(t+1)} &= \frac{ \sum_{i=1}^\nexp w_j^{(i)} + \sum_{i=1}^{\tilde{\nexp}}\tilde{w}_j^{(i)}} { \sum_{j=1}^k \left(\sum_{i=1}^\nexp w^{(i)}_j + \sum_{i=1}^{\tilde{\nexp}} \tilde{w}^{(i)}_j\right) } \\
    &= \frac{ \sum_{i=1}^\nexp w_j^{(i)} + \sum_{i=1}^{\tilde{\nexp}}\tilde{w}_j^{(i)}} { \nexp + \alpha \tilde{\nexp}}
    \end{align*}

    ${\mathbf \mu_j}$: Next, derive the update for $\mu_j$.  Do this by maximizing the expression with the argmax above with respect to $\mu_j$.\\

    First, calculate the gradient with respect to $\mu_j$:

    \begin{flalign*}
    \nabla_{\mu_j} &=
    % ### START CODE HERE ###
    % ### END CODE HERE ###
    \end{flalign*}

    Next, set the gradient to zero and solve for $\mu_j$:

    \begin{align*}
    0 &= \\
    % ### START CODE HERE ###
    % ### END CODE HERE ###
    \end{align*}

    ${\mathbf \Sigma_j}$: Finally, derive the update for $\Sigma_j$ via $S_j$.  Again, Do this by maximizing the expression with the argmax above with respect to $S_j$.\\.

    First, calculate the gradient with respect to $S_j$:

    \begin{flalign*}
    \nabla_{S_j} &= 
    % ### START CODE HERE ###
    % ### END CODE HERE ###
    \end{flalign*}

    Next, set the gradient to zero and solve for $S_j$:

    \begin{align*}
    0 &= \\
    % ### START CODE HERE ###
    % ### END CODE HERE ###
    \end{align*}

    This results in the final set of update expressions:
    \begin{align*}
      \phi_j & := \\
      % ### START CODE HERE ###
      % ### END CODE HERE ###
      \mu_j & :=  \\
      % ### START CODE HERE ###
      % ### END CODE HERE ###
      \Sigma_j & :=  \\
      % ### START CODE HERE ###
      % ### END CODE HERE ###
    \end{align*}
  \end{answer}
% <SCPD_SUBMISSION_TAG>_2c
\clearpage

\LARGE
2.f
\normalsize

% <SCPD_SUBMISSION_TAG>_2f
  \begin{answer}
    % ### START CODE HERE ###
    % ### END CODE HERE ###
  \end{answer}
% <SCPD_SUBMISSION_TAG>_2f
\clearpage

\LARGE
3.a
\normalsize

% <SCPD_SUBMISSION_TAG>_3a
  \begin{answer}
    % ### START CODE HERE ###
    % ### END CODE HERE ###
  \end{answer}
% <SCPD_SUBMISSION_TAG>_3a
\clearpage

\LARGE
3.b
\normalsize

% <SCPD_SUBMISSION_TAG>_3b
  \begin{answer}
    % ### START CODE HERE ###
    % ### END CODE HERE ###
  \end{answer}
% <SCPD_SUBMISSION_TAG>_3b
\clearpage

\LARGE
4.a
\normalsize

% <SCPD_SUBMISSION_TAG>_4a
  \begin{answer}
    % ### START CODE HERE ###
    % ### END CODE HERE ###
  \end{answer}
% <SCPD_SUBMISSION_TAG>_4a
\clearpage

\LARGE
4.b
\normalsize

% <SCPD_SUBMISSION_TAG>_4b
  \begin{answer}
    % ### START CODE HERE ###
    % ### END CODE HERE ###
  \end{answer}
% <SCPD_SUBMISSION_TAG>_4b
\clearpage

\LARGE
4.d
\normalsize

% <SCPD_SUBMISSION_TAG>_4d
  \begin{answer}
    % ### START CODE HERE ###
    % ### END CODE HERE ###
  \end{answer}
% <SCPD_SUBMISSION_TAG>_4d
\clearpage

\LARGE
4.e
\normalsize

% <SCPD_SUBMISSION_TAG>_4e
  \begin{answer}
    % ### START CODE HERE ###
    % ### END CODE HERE ###
  \end{answer}
% <SCPD_SUBMISSION_TAG>_4e
\clearpage

\LARGE
4.g
\normalsize

% <SCPD_SUBMISSION_TAG>_4g
  \begin{answer}
    % ### START CODE HERE ###
    % ### END CODE HERE ###
  \end{answer}
% <SCPD_SUBMISSION_TAG>_4g
\clearpage

\LARGE
4.i
\normalsize

% <SCPD_SUBMISSION_TAG>_4i
  \begin{answer}
    % ### START CODE HERE ###
    % ### END CODE HERE ###
  \end{answer}
% <SCPD_SUBMISSION_TAG>_4i
\clearpage


% <SCPD_SUBMISSION_TAG>_entire_submission

\end{document}